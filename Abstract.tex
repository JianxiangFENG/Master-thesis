\switchlanguage{en} % The abstract is supposed to be in English!

\thispagestyle{plain}

\section*{Abstract}

<<<<<<< HEAD
Deep learning based classifier has achieved tremendous success on different tasks ranging from vision, data mining to language processing and so on. However, this kind of classifier can't provide reliable uncertainty estimation about their predictions and expresses highly overconfidence which easily lead to serve consequences in safety-critical applications such as medical diagnosis, self-driving perception and so on. Especially in case of deployment of robots, the robot should be aware of its prediction in order to adapt itself to the environment with domain gap and avoid unnecessary accidents. Besides uncertainty of independent data example, uncertainty related to dependencies between data samples should be considered. When recognizing objects with similar appearances and different contexts in specific scene, discriminative classifier should express reliably high uncertainty about its prediction. At the same time, contextual information should be handled properly and utilized to resolve the ambiguity caused by similar appearances. 

In this work, Bayesian neural network including dropout variational inference and Laplace approximation is employed to improve uncertainty estimation. With better uncertainty estimation, robots can learn continuously by collecting dataset for fine-tuning with as little manual efforts as possible. Nevertheless, conditional random field is employed to improve the performance further by utilizing information from better uncertainty estimation and handling the dependencies between objects properly. Extensive experiments are performed to show that uncertainty estimation can be improved in terms of different evaluation metrics. Besides, experiments on two different datasets demonstrate that classifier can learn continuously to adapt itself in test environment which has domain gap to training set. Conditional random field is shown to be able to improve the performance further by making use of better uncertainty estimation and contextual information.
=======
Classifiers based on deep neural networks have achieved tremendous success on different kind of tasks. However, at the same time, this kind of classifier could not provide reliable uncertainty estimation about their predictions, which can easily lead to seriously overconfident predictions and sub-optimal decision in the down-stream tasks. From a systematic perspective, this is undesirable and even hazardous in wide range of safety-critical applications such as deployment of robots, medical diagnosis and so on. On the other hand, in object recognition task, some predictions are uncertain because they have similar appearances and thus ambiguous, although their relationship with contextual objects are distinguishable. For example a marker has similar appearance with a toothbrush, when a calculator is recognized nearby, a prediction of marker should have higher probability. This kind of uncertainty from data ambiguity can be considered in order to obtain a more robust classifier.  \maxcom{let's discuss about this part on tuesday :)}

In this work, regarding the first issue, we investigate applying dropout variational inference and Laplace approximation in Bayesian neural network to obtain reliable model uncertainty estimation on the object recognition task. Predictions with low uncertainty can be used to label observed data automatically, which can decrease manual effort on training a more accurate classifier when robot is confronted with objects in real world environment, which has a slight domain gap with objects in training set. When it comes to the second issue, object co-occurrence statistics is introduced as scene contextual information via conditional random field to handle the data ambiguity and improve the exiting results. We have conducted experiments to show that reliable uncertainty estimates can be obtained via dropout inference and even improved with proposed variants of dropout inference and its ensemble. Additionally, those improved uncertainty estimation can be used with semantic statistic to train a more accurate domain specific classifier.
>>>>>>> 7cbcb89a62caa7198db79c05b1e172ecbfb40e52

\switchlanguage{\lang} % Switch back to the document's default language.
