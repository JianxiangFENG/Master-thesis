\switchlanguage{en} % The abstract is supposed to be in English!

\thispagestyle{plain}

\section*{Abstract}
Deep learning based classifier has achieved tremendous success on different tasks. However, this kind of classifier can't provide reliable uncertainty estimation about their predictions and expresses highly overconfidence which easily lead to serve consequences in safety-critical applications such as medical diagnosis, perception of self-driving and so on. Especially in case of deployment of robots, the robots should be aware of its prediction in order to adapt itself to the environment in real world and avoid unnecessary accidents. Besides uncertainty of independent data example, uncertainty related to dependencies between data samples should be considered. When recognizing objects with similar appearances but with different contexts in specific scene, discriminative classifier should express reliably high uncertainty about its prediction because of similar appearance. At the same time, contextual information should be handled properly and utilized to resolve the ambiguity caused by similar appearances. 

In this work, Bayesian neural network with inference techniques such as dropout variational inference and scalable Laplace approximation is employed to improve uncertainty estimation. With better uncertainty estimation, robots can learn continuously by collecting dataset for fine-tuning with as little manual efforts as possible. Nevertheless, conditional random field is employed to improve the performance further by utilizing more information brought by better uncertainty estimation and handling the dependencies between objects properly. Extensive experiments are performed to show that uncertainty estimation can be improved with Bayesian neural network in terms of different evaluation metrics. Besides, experiments on two different datasets demonstrate that classifier are capable to learn continuously to adapt itself in test environment which has domain gap to training set with help of improved uncertainty estimation. Last but not least, experiments also show that conditional random field is able to improve the performance further by making use of better uncertainty estimation and contextual information.

\switchlanguage{\lang} % Switch back to the document's default language.
