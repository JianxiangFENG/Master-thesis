\chapter{Appendix}
% \section{KL condition}
% \label{appendix:kl_condition}
% \paragraph{Proposition} Fix $K$, $L \in \mathbb N$, a probability vector $\bld p = (p_{1}, ..., p_{L})$, and $\bld \Sigma_{i} \in \mathbb R^{K \times K}$ diagonal positive definite for $i=1,...,L$, with the elements of each $\bld \Sigma_{i}$ not dependent on $K$. Let
%\[
%q(\bld x) = \sum_{i=1}^{L} p_{i}\mathcal N(\bld x; \bld \mu_{i}, \bld \Sigma_{i}) 
%\]
%be a mixture of Gaussians with $L$ components and $\bld \mu_{i} \in \mathbb R^{K}$, let $p(\bld x) = \mathcal N(0, \bld I_{K})$, and further assume that $\bld \mu_{i} - \bld \mu_{j} \sim \mathcal N(0, \bld I)$ for all $i$ and $j$.
%
%The KL divergence between $q(\bld x)$ and $p(\bld x)$ can be approximated as:
%\[ 
%KL(q(\bld x) || p(\bld x)) \approx \sum_{i=1}^{L}\frac{p_{i}}{2}(\bld \mu_{i}^{T} \bld \mu_{i} + tr(\bld \Sigma_{i}) - K(1+\text{log}2\pi) - \text{log}(\text{det}(\bld \Sigma_{i}))) - \mathcal H (\bld p)
%\] 
%with 
%\[
%\mathcal H(\bld p) = -\sum_{i=1}^{L}p_{i}\text{log}p_{i}
%\]
%for large enough $K$. 
%
%\paragraph{$Proof.$}
%We have 
%\[
%\begin{aligned}
%KL(q(\bld x) || p(\bld x)) &= \int q(\bld x) \text{log }\frac{q(\bld x)}{p(\bld x)} d\bld x \\
%&=\int q(\bld x) \text{log} q(\bld x) d\bld x - \int q(\bld x)\text{log} p(\bld x) d\bld x \\
%&=-\mathcal H(q(\bld x)) - \int q(\bld x)\text{log } p(\bld x) d\bld x 
%\end{aligned}
%\]
%which is sum of entropy of $q(\bld x)$ and the expected log probability of $\bld x$. The expected log probability can be evaluated analytically, but the entropy term has to be approximated.
%
%We begin by approximating the entropy term. We write 
%\[
%\begin{aligned}
%\mathcal H(q(\bld x)) &= -\sum_{i=1}^{L}p_{i}\int \mathcal N(\bld x; \bld \mu_{i}, \bld \Sigma_{i})\text{log }q(\bld x)d\bld x \\
%&=-\sum_{i=1}^{L} p_{i}\int \mathcal N(\bld \epsilon_{i}; 0, \bld I) \text{log } q(\bld \mu_{i} + \bld L_{i} \bld \epsilon_{i}) d\bld \epsilon_{i} \\ 
%\end{aligned}
%\] 
%with a re-parameterization of $\bld x$, that is $\bld x = \bld \mu_{i} + \bld L_{i} \bld \epsilon_{i}$ with $\bld L_{i} \bld L_{i}^{T} = \bld \Sigma_{i}$ and $\bld \epsilon_{i} \sim \mathcal N(0, I)$.
%
%Now the term inside logarithm can be written as 
%\[
%\begin{aligned}
%q(\bld \mu_{i} + \bld L_{i}\bld \epsilon_{i}) &= \sum_{j=1}^{L} p_{j} \mathcal N(\bld \mu_{i}+\bld L_{i}\bld \epsilon_{i}; \bld \mu_{j}, \bld \Sigma_{j}) \\
%&=\sum_{j=1}^{L}p_{j}(2\pi)^{-\frac{K}{2}} \text{det}(\bld \Sigma_{j})^{-\frac{1}{2}}\text{exp }\{-\frac{1}{2}||\bld \mu_{j} - \bld \mu_{i} - \bld L_{i} \bld \epsilon_{i}||^{2}_{\bld \Sigma_{j}}\}
%\end{aligned}
%\]
%where $||\cdot||_{\bld \Sigma}$ is Mahalanobis distance with $\bld \Sigma$ as covariance matrix. Since $\bld \mu_{i} - \bld \mu_{j}$ is assumed to be normally distributed, the quantity $\bld \mu_i - \bld \mu_{j} - \bld L_{i} \bld \epsilon_{i} $ is also normally distributed. Since the expectation of a generalized $\mathcal X^{2}$ distribution with $K$ degrees of freedom increases with $K$, we have that $K \gg 0$ implies that $||\bld \mu_i - \bld \mu_{j} - \bld L_{i} \bld \epsilon_{i} ||^{2}_{\bld \Sigma_{j}} \gg 0 $ for $i \neq j$(since the elements of $\bld \Sigma_{j}$ does not depend on $K$). Finally, we have for $i=j$ that  $||\bld \mu_i - \bld \mu_{j} - \bld L_{i} \bld \epsilon_{i} ||^{2}_{\bld \Sigma_{j}} = \bld \epsilon_{i}^{T} \bld L_{i}^{T} \bld L_{i}^{-T} \bld L_{i}^{-1} \bld L_{i} \bld \epsilon_{i} = \bld \epsilon_{i}^{T} \bld \epsilon_{i}$. Therefore the last equation can be approximated as 
%\[
%q(\bld \mu_{i} + \bld L_{i} \bld \epsilon_{i}) \approx p_{i}(2\pi)^{-\frac{K}{2}}\text{det}(\bld \Sigma_{i})^{-\frac{1}{2}}\text{exp }\{-\frac{1}{2}\bld \epsilon_{i}^{T} \bld \epsilon_{i}\}
%\]
%I.e. in high dimensions the mixture components will not overlap.This gives us 
%\[
%\begin{aligned}
%\mathcal H(q(\bld x)) & \approx -\sum_{i=1}^{L} p_{i}\int \mathcal N(\bld \epsilon_{i}; 0, \bld I) \text{log } \big(p_{i}(2\pi)^{-\frac{K}{2}}\text{det}(\bld \Sigma_{i})^{-\frac{1}{2}}\text{exp }\{-\frac{1}{2}\bld \epsilon_{i}^{T} \bld \epsilon_{i}\}\big) d\bld \epsilon_{i} \\ 
%&= -\sum_{i=1}^{L}p_{i}\big[\text{log }p_{i} - \frac{K}{2}\text{log }2\pi  - \frac{1}{2} \text{log }\text{det}(\bld \Sigma_{i}) - \frac{1}{2}\int \mathcal N(\bld \epsilon_{i}; 0, \bld I)\bld \epsilon_{i}^{T} \bld \epsilon_{i} d \bld \epsilon_{i}\big] \\
%&=\sum_{i=1}^{L}\frac{p_{i}}{2}\big(K\text{log }2\pi + \text{log } \text{det}(\bld \Sigma_{i}) + \int \mathcal N(\bld \epsilon_{i};0, \bld I)\bld \epsilon_{i}^{T}\bld \epsilon_{i}d\bld \epsilon_{i} \big) + \mathcal H(\bld p)
%\end{aligned}
%\] 
%Since $\bld \epsilon_{i}^{T}\bld \epsilon_{i}$ distributes according to a $\mathcal X^{2}$ distribution , its expectation is $K$, and in the end the entropy term can be approximated as 
%\[
%\mathcal H(q(\bld x)) \approx \sum_{i=1}^{L}\frac{p_{i}}{2}\big(K(\text{log }2\pi + 1) + \text{log } \text{det}(\bld \Sigma_{i}) \big) + \mathcal H(\bld p)
%\]
%
%Next, we can evaluate the expected log probability term, we get 
%\[
%\int q(\bld x) \text{log }p(\bld x) d\bld x = \sum_{i=1}^{L}p_{i}\int \mathcal N(\bld x; \bld \mu_{i}, \bld \Sigma_{i})\text{log }p(\bld x) d\bld x
%\]
%for $p(\bld x) = \mathcal N(0, \bld I_{K})$, it is easy to show that
%\[
%\begin{aligned}
%\int q(\bld x) \text{log }p(\bld x) d\bld x &= \sum_{i=1}^{L}p_{i} \int 
%\mathcal N(\bld x; \bld \mu_{i}, \bld \Sigma_{i}) \text{log}\big[(2\pi)^{-\frac{K}{2}} \text{det}(\bld I_{K})^{-\frac{1}{2}} \text{exp}\{-\frac{1}{2}\bld x^{T} \bld x \} \big] d\bld x \\ 
%&= \sum_{i=1}^{L}p_{i} \big[
%-\frac{K}{2} \text{log} 2\pi
%-\frac{1}{2}\int \mathcal N(\bld x; \bld \mu_{i}, \bld \Sigma_{i}) \bld x^{T} \bld x d\bld x
%\big]\\
%&=-\frac{1}{2}\sum_{i=1}^{L}p_{i}(\bld \mu_{i}^{T} \bld \mu_{i} + tr(\bld \Sigma_{i}) + K\text{log}2\pi)
%\end{aligned}
%\]
%
%Finally, combining the equations above, we have 
%\[ 
%KL(q(\bld x) || p(\bld x)) \approx \sum_{i=1}^{L}\frac{p_{i}}{2}(\bld \mu_{i}^{T} \bld \mu_{i} + tr(\bld \Sigma_{i}) - K - \text{log}(\text{det}(\bld \Sigma_{i}))) - \mathcal H (\bld p)
%\] 
%as required to show.  
%\begin{flushright}
%	$\square$
%\end{flushright}

\section{Fisher information} \label{appendix:fisher_matrix}
Here, we consider the case that parameter of likelihood function is scalar. This can easily be extended to the case that $\theta$ is a vector. Fisher information $\mathcal{I}(\theta)$ is defined as follows:
\begin{equation}
	\mathcal{I}(\theta) = \mathbb{E}\big[\big(\frac{\partial}{\partial \theta}\log{f(X; \theta)}\big)^2\big]
\end{equation}
Where $f(X;\theta)$ is probability density function or mass function of random variable $X$. We need to prove that $\mathbb{E}\big[\big(\frac{\partial}{\partial \theta}\log{f(X; \theta)}\big)^2\big] = \mathbb{E}\big[-\frac{\partial^2}{\partial \theta^2}\log{f(X; \theta)}\big]$.
When we suppose that $\theta$ is the true parameter. Based on the fact that $f(X;\theta)$ is density, we can have:
\[
\int f(x;\theta) dx = 1
\]
Differentiating the above with respect to $\theta$ gives:
\[
\frac{\partial}{\partial \theta}\int f(x;\theta) dx = 0
\]
After moving differentiating operation into integration and making use of derivative of logarithm, we get:
\[
\int \frac{\partial \log f(x;\theta)}{\partial \theta} f(x;\theta) dx = 0
\]
Differentiating again w.r.t. $\theta$ and taking the derivative inside gives:
\[
\begin{aligned}
&\int \frac{\partial^2\log f(x;\theta)}{\partial \theta^2}f(x;\theta) dx + \int \frac{\partial\log f(x;\theta)}{\partial \theta} \frac{\partial f(x;\theta)}{\partial \theta}dx
= 0\\
\Rightarrow &\int \frac{\partial^2\log f(x;\theta)}{\partial \theta^2}f(x;\theta) dx + \int \frac{\partial\log f(x;\theta)}{\partial \theta} \frac{\partial f(x;\theta)}{\partial \theta}\frac{1}{f(x;\theta)} f(y;\theta) dx= 0 \\
\Rightarrow &\int \frac{\partial^2\log f(x;\theta)}{\partial \theta^2}f(x;\theta) dx + \int \frac{\partial\log f(x;\theta)}{\partial \theta} \frac{\partial \log f(x;\theta)}{\partial \theta} f(y;\theta) dx= 0 \\
\Rightarrow & -\mathbb{E}[\frac{\partial^2\log f(x;\theta)}{\partial \theta^2}] = \mathbb{E}[\frac{\partial\log f(x;\theta)}{\partial \theta} \frac{\partial \log f(x;\theta)}{\partial \theta} ] \\
\end{aligned}
\]
which gives us the required result.
 \begin{flushright}
 	$\square$
 \end{flushright}
 
\section{Derivation of Re-parameterization Trick}\label{d_repa}
Let $p_\theta(\bld x)$ denote the density function of $\bld x$ parameterized by $\theta$, $g(\cdot)$ is a differential function, $\epsilon$ is a random variable with parameter-free density function $p(\epsilon)$. The estimator of the gradients w.r.t. $\theta$ with $p(x|\epsilon) = \delta(x-g(\theta,\epsilon))$ can be derived as:
\begin{equation}
\begin{aligned}
I'(\theta) &= \frac{\partial}{\partial \theta} \int f(x) p_{\theta}(x) dx \\
&= \frac{\partial}{\partial \theta} \int f(x)(\int p_{\theta}(x|\epsilon)p(\epsilon)d\epsilon) dx \\
&= \frac{\partial}{\partial \theta} \int (\int f(x) p_{\theta}(x|\epsilon)dx) p(\epsilon) d\epsilon \\
&= \frac{\partial}{\partial \theta} \int (\int f(x)\delta(x-g(\theta,\epsilon))dx) p(\epsilon) d\epsilon \\ 
&= \frac{\partial}{\partial \theta} \int f(g(\epsilon, \theta)) p(\epsilon) d\epsilon \\
&= \int \frac{\partial}{\partial \theta} f(g(\epsilon, \theta)) p(\epsilon) d\epsilon \\
&= \int f'(g(\epsilon, \theta))\frac{\partial}{\partial \theta}g(\theta, \epsilon) p(\epsilon) d\epsilon \\
&= \mathbb E_{p(\epsilon)}\big[ f'(g(\epsilon, \theta))\frac{\partial}{\partial \theta}g(\theta, \epsilon)\big] 
\end{aligned}
\end{equation} 
 \begin{flushright}
	$\square$
\end{flushright}
\section{Second Order Taylor Expansion of Log Posterior}\label{d_second_Hessian}
$p(\bld \omega| \bld Y, \bld X)$ denotes the posterior over model parameters $\bld \omega$. $\bld \omega^\star$ is one optimizer obtained after training. Then the second order Taylor expansion of posterior distribution around $\bld \omega^\star$ is defined as follows:
 \begin{equation}
\begin{aligned} 
\text{log} p(\bld \omega| \bld Y, \bld X) \approx & 
\text{log}p(\bld \omega^\star|\bld Y, \bld X) + \\
&(\bld \omega - \bld \omega^\star)^T\frac{\partial \text{log} p(\bld \omega| \bld Y, \bld X)}{\partial \bld \omega} + \\
&\frac{1}{2}(\bld \omega - \bld \omega^\star)^T\frac{\partial^2\text{log} p(\bld \omega |\bld Y, \bld X)}{\partial \bld \omega^2}(\bld \omega - \bld \omega^\star)\\
= &\text{log}p(\bld \omega^\star|\bld Y, \bld X) - \frac{1}{2}(\bld \omega - \bld \omega^\star)^T\bld H(\bld \omega - \bld \omega^\star)
\end{aligned}
\end{equation}
where the term of first derivatives vanishes because the gradients around an optimizer is zero and 
\[
\begin{aligned}
\bld H &= -\frac{\partial^2\text{log} p(\bld \omega|\bld Y, \bld X)}{\partial \bld \omega^2}\\
&=-\frac{\partial^2\text{log}\big(p(\bld Y|\bld X, \bld \omega)\big)}{\partial \bld \omega^2} - \frac{\partial^2p(\bld \omega)}{\partial\bld \omega^2}
\end{aligned}
\]

, where $p(\bld Y | \bld X, \bld \omega)$ is the likelihood function, $p(\bld \omega)$ is the prior distribution over model parameters.
 \begin{flushright}
	$\square$
\end{flushright}
%\section{Implementation details}


